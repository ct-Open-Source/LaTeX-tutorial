% main.tex
\documentclass{scrlttr2} % KOMA-Script-Briefklasse

% --- Umschaltbarer Ton für die Schlussformel ---
% true  = formaler Ton, false = freundschaftlicher Ton
\newif\ifformal
\formaltrue % oder: \formalfalse

% --- Einbinden der Layout-Datei (.lco) ---
\LoadLetterOption{letter}

% --- PDF-Metadaten und klickbare Links ---
\usepackage[pdfauthor={Bilbo Beutlin},
            pdftitle={Einladung zum Tee},
            pdfsubject={Persönlicher Brief}]{hyperref}

% --- Kundennummer. Weitere Optionen: yourref, myref, yourmail, invoice ---
\setkomavar{customer}{00001}
\setkomavar*{customer}{Magier-Nr.} % Default: Kundennummer

% --- Ort und Datum ---
\setkomavar{place}{Hobbingen}
\setkomavar{date}[Datum A.Z.]{\today} %Default: Datum

% --- Betreff ---
\setkomavar{subject}{Einladung zum Tee}

% --- Brief ---
\begin{document}

\begin{letter}{% Adressat
Gandalf der Graue\\
Irgendwo in Mittelerde}

\opening{Lieber Gandalf,}

% Inhalt des Briefs
hiermit lade ich dich herzlich zu einem gemütlichen Tee ein.
Bitte bring Zeit und gute Geschichten mit.

\lipsum[2-4] % Blindtext

\closing{\myclosing} % Schlussformel und Unterschrift

\encl{Kuchenliste, Fotos aus Hobbingen} % Anlagen

\ps{PS: Bitte bring Lembas mit!} % PS

\end{letter}

\end{document}